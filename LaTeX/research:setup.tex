\section{Setup}

\subsection{wpa\_supplicant}

The implementation we used for our research is the wpa\_supplicant\footnote{https://w1.fi/} software written by Jouni Malinen. This Linux user space 802.11 client is used in all major Linux distributions and mobile operating system Android.
Along with wpa\_supplicant comes the access point software called hostapd. We do not look at it's implementation but it will serve as virtual access point during our testing of wpa\_supplicant. Our research could be conducted with any type of access point.

We will use version 2.7 of hostapd and wpa\_supplicant to conduct our research. This version was released in December 2018.

\subsection{Networking}

Before we can run the mapper and learner to infer the state machine we'll have to setup a simulated network environment. Since the wpa\_supplicant/hostapd source code already includes automated tests that use such a simulated environment we'll be partially reusing this implementation.
The environment requires a special build of both wpa\_supplicant and hostapd which can be constructed by following the instructions in the repository \cite{Malinen:2013}.
Our test will use the mac80211\_hwsim Linux kernel driver which is capable of simulating 802.11 hardware. We setup three interfaces: one WPA2 protected access point and two stations. The access point is running hostapd, the stations are controlled by wpa\_supplicant. Since we know the MAC addresses of the access point and both stations we're able to craft messages and send them via the AP to the receiving wpa\_supplicant instance. The response of that instance can then be read by sniffing the interface of the interface we simulated to be the sender.

\begin{figure}[!h]
	\centering
	\includegraphics[width=\textwidth]{setup}
	\caption{A visual representation of our setup}
	\label{fig:setup}
\end{figure}

\subsection{Learner settings}

As we mentioned previously (\ref{preliminaries:learning:process}) we'll use the L* and randomwords algorithms for the learning process. We will setup the equivalence algorithm to use a minimum and maximum length of 5 and 10 respectively. We will require 5000 queries to prove that no counterexample can be found.